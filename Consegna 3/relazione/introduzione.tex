\section{Introduzione}

è valutare le prestazioni dell'algoritmo di Karger per il problema del minimum cut rispetto a quattro parametri:
\begin{itemize}
\item Il tempo impiegato dalla procedura di Full Contraction
\item Il tempo impiegato dall'algoritmo completo per ripetere la contrazione un numero sufficientemente alto di volte
\item Il discovery time, ossia il momento in cui l'algoritmo trova per la prima volta il taglio di costo mimimo
\item L'errore nella soluzione trovata rispetto al risultato ottimo
\end{itemize}
Il linguaggio di programmazione scelto dal nostro gruppo è Java.

\subsection{Esecuzione del programma}
Gli algoritmi sono stati sviluppati come progetto Maven. All'interno della cartella \'e presente la versione portable di Maven, pertanto non è necessario averlo installato. \'E
richiesto almeno il JDK 11 installato nel sistema.
Per eseguire i tre algoritmi utilizzare i seguenti comandi:\\
Linux:\\
\indent \texttt{./mvnw install}\\
\indent \texttt{./mvnw exec:java}\\
Windows:\\
\indent \texttt{mvnw.cmd install}\\
\indent \texttt{mvnw.cmd exec:java}

L'esecuzione del main genera automaticamente dei file csv nella directory del progetto contenenti i tempi registrati. 
L'algoritmo di Karger viene interrotto dopo 60 secondi.
\subsection{Strutture dati utilizzate}

Per rappresentare il grafo abbiamo utilizzato una lista di adiacenza, quindi ogni nodo contiene la lista dei lati adiacenti ad esso.
Un lato viene trattato come un'istanza della classe \texttt{Edge}.

\subsection{Lettura di un grafo da file}
Per caricare un grafo in memoria, abbiamo implementato una classe GraphReader, 
che si occupa della lettura del file tramite la libreria \textit{nio} di Java. 
La classe infine ritorna un'istanza della classe \texttt{Graph}.
\subsection{Implementazione di Karger}
L'algoritmo di Karger contiene un metodo \texttt{solve} che ha come parametri: il grafo, 
il numero di ripetizioni da effettuare, e infine il timeout, ovvero la durata massima 
dell'esecuzione dell'algoritmo.
Per eseguire le misurazioni dei tempi di esecuzioni si fa uso dell'istruzione 
\texttt{System.currentTimeMillis()} che ritorna il tempo attuale in ms. 
Essendo il programma single thread non c'è il rischio che l'esecuzione venga interrotta e 
quindi permette di ottenere misure affidabili. 
Infine, per quanto riguarda la codifica l'algoritmo si occupa di richiamare \texttt{k}-volte
il metodo Full-Contraction calcolando il valore del taglio minimo.
\subsubsection{Implementazione del metodo Full Contraction}