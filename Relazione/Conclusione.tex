\section{Confronto}
Sicuramente l'algoritmo con costo computazionale più basso è Kruskal con Union-Find.
Confrontiamo quindi l'esecuzione di Kruskal con Union-Find con gli altri due algoritmi partendo da Prim. 
\begin{figure}[H]
\centering
\includegraphics[scale=0.5]{grafici/primvskruskaluf.pdf}
\caption{Numero di grafi di dimensione $n$ che Kruskal con Union-Find è in grado di elaborare nello stesso tempo medio impiegato da Prim per elaborare \textbf{un} grafo  della medesima dimensione. Come possiamo vedere Kruskal è molto più efficiente in grafi di grandi dimensioni.}
\end{figure}
Confrontiamo ora l'algoritmo di Kruskal con Union-Find con la sua versione naive:
\begin{figure}[H]
\centering
\includegraphics[scale=0.5]{grafici/kruskalvskruskaluf.pdf}
\caption{Numero di grafi di dimensione $n$ che Kruskal con Union-Find è in grado di elaborare nello stesso tempo medio impiegato da Kruskal per elaborare \textbf{un} grafo della medesima dimensione. Come possiamo vedere Kruskal è molto più efficiente in grafi di grandi dimensioni.}
\end{figure}
Da questi due grafici e nella tabella \ref{t1} possiamo affermare che Kruskal con Union-Find è l'algoritmo \textbf{più} efficiente per il calcolo di un MST, segue l'algoritmo di Prim, e infine Kruskal "Naive" che ha il costo computazionale più alto.
\section{Conclusione}